\documentclass[12pt,a4paper]{article}

% Packages
\usepackage[utf8]{inputenc}
\usepackage{geometry}
\geometry{margin=1in}
\usepackage{graphicx}
\usepackage{amsmath}
\usepackage{cite}

\title{Literature Review: Bolted vs. Welded Moment Connections}
\author{Prateeksha Sharma}
\date{}

\begin{document}

\maketitle

\section{Introduction}
Moment-resisting connections are a critical component of steel frame structures, providing rigidity and the ability to transfer both shear and bending moments between connected members. Two most commonly used moment connections are bolted moment connections and welded moment connections, and each of these connections affects the structural performance, cost and durability in its own way.

\section{Structural Performance}
Welded moment connections provide higher stiffness resulting in a sort of monolithic behavior between the connected members. This is due to the absence of mechanical slip. This is useful in high seismic as well as high wind situations. Comparable strength can be achieved even by the bolted connections when they use high friction grip.

The experiments have shown that bolted end-plate connections can have comparable moment capacity to welded flange connections, provided they are properly detailed, and bolt pretensioning and plate thickness is sufficient.
 The weld connections can have reduced moment capacity if there are to weld defects or residual stresses due to lack of quality control.

\section{Fabrication and Erection}
The key difference between the connection types is the fabrication. 
Welded connections require skilled labor, controlled conditions of welding, preheating and post-weld cooling in order to avoid residual stresses and distortions. Bolted connections can be installed in controlled environment and are generally pre-fabricated like bolted flange plates or extended end-plates.
Welded connections are adversely affected by the weather conditions and temperature variations. Bolted connections are more resistant to the weather and temperature variations.

For inspection, welded joints need ultrasonic or radiographic testing to detect internal flaws. Bolted connections require just a visual inspection generally, and torque verification.
For maintenance, welded connections require more invasive and complex procedures while the bolted connections are easier to disassemble or replace without cutting or grinding.


\section{Cost}
Cost depends heavily on the scale of the project and labor rates. Welded connections have higher initial labor costs due to skilled welding requirements, but have less number of  fabricated components. Bolted connections have lower labor costs but require additional fabrication time for drilling, plate cutting, and bolt preparation. The bolted systems may offer lower total costs if we consider erection speed and reduced labor, especially in prefabricated projects.

\section{Seismic and Fatigue Behavior}
In the seismic structures, welded flange-bolted web connections were widely used before the 1994 Northridge earthquake, where brittle fractures in welded joints rendered welded joints insufficient. Post-Northridge research demonstrated that bolted end-plate and bolted flange plate systems, with proper design, can dissipate energy effectively while avoiding some weld-related brittle failure modes.

In fatigue-sensitive structures like bridges, bolted connections may be preferred by minimizing residual stresses over welding, though there may be bolt preload loss over time.


\section{Conclusion}
The choice between bolted and welded moment connections depends on the project, and depend upon structural demands, inspection requirements, and budget. Recent research favors combining bolted and welded elements to optimize performance, reduce fabrication complexity, and improve seismic resilience. 
\begin{thebibliography}{99}

Geschwindner, J. (2000). Unified Design of Steel Structures (2nd ed.). John Wiley & Sons.


\end{thebibliography}

\end{document}
